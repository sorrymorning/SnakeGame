\documentclass{article}
\usepackage{enumitem}
\usepackage{hyperref}

\begin{document}

\title{Documentation BrickGame v1.0 aka Tetris}
\date{}
\maketitle

\section{Introduction}

Project BrickGame v1.0 aka Tetris is an implementation of the classic arcade game Tetris using the C programming language and the ncurses library for the terminal interface.

\section{Project structure}

\subsection{Tetris library (\texttt{src/brick\_game/tetris})}

\begin{itemize}[label=--]
    \item Source code files implementing the logic of the Tetris game.
    \item Core functions for interacting with the game field, controlling pieces, and handling user input.
    \item Implementation of a finite state machine to formalize the game logic.
\end{itemize}

\subsection{Terminal interface (\texttt{src/gui/cli})}

\begin{itemize}[label=--]
    \item Source code files responsible for rendering the game in the terminal using the ncurses library.
    \item Implementation of game field rendering, user input handling, and displaying the current game state.
\end{itemize}

\section{Project build}

The project uses the \texttt{make} build system with a Makefile that includes the following targets:

\begin{itemize}
    \item \texttt{all}: Build the project.
    \item \texttt{install}: Install the program into the system.
    \item \texttt{uninstall}: Remove the program from the system.
    \item \texttt{clean}: Remove temporary files and folders.
    \item \texttt{dvi}: Create a DVI file.
    \item \texttt{dist}: Create an archive containing the necessary files for building and running the program.
\end{itemize}

\section{Runtime requirements}

The project requires the C11 programming language, the gcc compiler, and the ncurses library for the terminal interface.

\section{Testing}

The project includes unit tests using the \texttt{check} library. The test coverage for the library is at least 80\%.

\end{document}
